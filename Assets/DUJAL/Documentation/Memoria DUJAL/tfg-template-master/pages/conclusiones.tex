\section{Consecución de Objetivos}

Al inicio de este documento se enumeraba un resumen de los objetivos generales y secundarios de este proyecto, que incluían.

Principales:
En resumen, el el proyecto pretende, como objetivo principal desarrollar un paquete de herramientas completo y funcional que permita a los usuarios prototipar y 
 desarrollar videojuegos. Dicho paquete debe contener diversos sistemas y componentes 'prefabricados' que agilicen el desarrollo. 

\begin{compactitem}
    \item Desarrollar un paquete de herramientas completo y funcional.
    \item Dicho paquete debe permitir a los usuarios prototipar y desarrollar videojuegos.
    \item Tambén debe contener una serie de módulos divididos en: Sistemas Atómicos, Herramientas de Debug, Generador de Mazmorras, Componentes Independientes y Componentes de Movimiento de Personajes.
\end{compactitem}

Secundarios:
\begin{compactitem}
  \item Que los componentes del paquete sean escalables.
  \item Desarrollar niveles de prueba para servir a modo de tutorial para los usuarios.
  \item Documentar el modo de uso y funcionamiento de los componentes del proyecto.
  \item Validar el paquete con usuarios reales.
  \item Construir varios prototipos utilizando el paquete.
  \item Ajustar el paquete en base a los resultados de una encuesta de satisfacción.
\end{compactitem}

A continuación se discute el grado de consecución de cada objetivo:
\begin{enumerate}[itemsep=0mm]

\item Conclusion objetivo 1.
  
\item Conclusion objetivo 2.

\item Conclusion objetivo 3.

\end{enumerate}

\section{Trabajo Futuro}

De cara a futuras actualizaciones de DUJAL y teniendo en cuenta el feedback, habría que añadir una serie de mejoras: 
\begin{itemize}
    \item Mejora 1.
    \item Mejora 2.
    \item Mejora 3.
\end{itemize}

\section{Conclusiones Personales}
Personalmente opino que el proyecto ha sido un rotundo éxito, se ha desarrollado un paquete que a todas luces ha resultado verdaderamente útil de cara al prototipado de
 videojuegos con componentes que además de ser usables e intuitivos, el feedback ha demostrado que también son escalables. Además, a un nivel personal he aprendido mucho acerca
  de las tecnologías acerca que las que quería aprender, y estoy orgulloso del resultado tanto a nivel de usuario como con la complejidad técnica de la que se compone el proyecto. 