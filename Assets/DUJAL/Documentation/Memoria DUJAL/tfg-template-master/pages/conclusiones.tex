\section{Consecución de Objetivos}

Al inicio de este documento se enumeraba un resumen de los objetivos generales y secundarios de este proyecto, que incluían.

Principales:
En este proyecto se pretende desarrollar un paquete de herramientas completo y funcional que permita a los usuarios prototipar y desarrollar videojuegos. Dicho paquete debe contener diversos sistemas y
 componentes 'prefabricados' que agilicen el desarrollo. 

\begin{compactitem}
    \item Desarrollar un paquete de herramientas completo y funcional.
    \item Dicho paquete debe permitir a los usuarios prototipar y desarrollar videojuegos.
    \item Tambén debe contener una serie de módulos divididos en: Sistemas Atómicos, Herramientas de Debug, Generador de Mazmorras, Componentes Independientes y Componentes de Movimiento de Personajes.
\end{compactitem}

Secundarios:
\begin{compactitem}
  \item Que los componentes del paquete sean escalables.
  \item Desarrollar niveles de prueba para servir a modo de tutorial para los usuarios.
  \item Documentar el modo de uso y funcionamiento de los componentes del proyecto.
  \item Validar el paquete con usuarios reales.
  \item Construir varios prototipos utilizando el paquete.
  \item Ajustar el paquete en base a los resultados de una encuesta de satisfacción.
\end{compactitem}

A continuación se discute el grado de consecución de cada objetivo:
\begin{enumerate}[itemsep=0mm]

\item El objetivo principal queda completado sin lugar a dudas, se ha construido una librería completamente funcional que ha ayudado en el desarrollo de dos proyectos formales y varios prototipos cortos, dichos 
proyectos han sido desarrollados por una serie de encuestados que en su mayoría concluyen que la libería ha sido una ayuda para su trabajo.
  
\item Se han podido completar todos los módulos anticipados al principio del proyecto, incluyendo todos los sub-módulos indicados en la introducción \ref{sec:intro}.

\item Dados los resutlados de la validación, queda claro que aquellos encuestados con perfil técnico opinan que el código es escalable y fácil de entender, permitiendo añadir nuevas funcionalidades o adapter 
las que puedan no ajustarse del todo a un proyecto dado.

\item Se han desarrollado séis niveles de prueba que permiten al usuario ver ejemplificado el uso de los distintos componentes de la librería. Sin embargo, parte del feedback indica que la documentación es 
mejorable, y es en parte por que la explicación de dichos niveles de ejemplo no es del todo completa.

\item Se ha desarrollado un documento indicando el funcionamiento y tecnicismos de todos los componentes de la librería, como se ha mencionado en el punto anterior, el feedback de un encuestado mencionaba que
 es mejorable.
 
\item El paquete se ha validado satisfactoriamente frente a un gran número de usuarios finales, estos han sido participativos en el proceso y han desarrollado prototipos y proyectos que han sido un exito en 
sus respectivos casos.

\item El paquete se ha ajustado ligeramente para reflejar el feedback de los usuarios. Se ha desarrollado una mejora para el paquete SnapToGrid y se planea extender la documentación para incluir más detalles 
y ejemplos de uso, dicha mejora de la documentación debería paliar también el feedback que indica que ciertos componentes no son sencillos para perfiles alejados de lo técnico.

\end{enumerate}

\section{Trabajo Futuro}

De cara a futuras actualizaciones de DUJAL y teniendo en cuenta el feedback, habría que añadir una serie de mejoras: 
\begin{itemize}
    \item La previamente mencionada mejora de la documentación, es importante que el usuario pueda ver ejemplos y fotos que expliquen como utilizar las herramientas que posee.
    \item Sería interesante añadir interfaces gráficas como la del sistema de diálogo para otros componentes, de forma que permita a los usuarios editar las configuraciones de los distintos módulos 
    de forma más visual.
\end{itemize}

\section{Conclusiones Personales}
Personalmente opino que el proyecto ha sido un rotundo éxito, se ha desarrollado un paquete que a todas luces ha resultado verdaderamente útil de cara al prototipado de
 videojuegos con componentes que además de ser usables e intuitivos en su mayoría, el feedback ha demostrado que también son escalables. Además, a un nivel personal he aprendido mucho acerca
  de las tecnologías acerca que las que me había propuesto aprender, y estoy orgulloso del resultado tanto a nivel de usuario como con la complejidad técnica de la que se compone el proyecto. 